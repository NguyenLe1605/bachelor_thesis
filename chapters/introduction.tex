\chapter{Introduction}
\section{Motivation}
  As VPN is mentioned in many papers as not suitable for constrained small IoT Devices, 
  I start to question, is it really the true for in this modern area. 
  As a modern network protocols called wireguard show up, with an aim to replace
  the old, sound but complex protocols like IPSec, we here consider adapting the wireguard
  protocols onto this constrained IoT environment, to see whether this VPN protocol is 
  a good fit for the IoT world or not.

  WireGuard [23] is a new VPN protocol that fits the role of this new pipe
and it looks quite promising. Note that WireGuard was originally presented
at NDSS 2017 [15], but while the main concepts still apply, the protocol has
slightly evolved in an incompatible way. The latest version is described in
the WireGuard whitepaper [18].
WireGuard is designed using modern cryptography, aims for high per-
formance and reduces the attack surface as a simple protocol. Unlike other
protocols, no form of negotiation over cryptographic parameters is possible.
Instead, it uses the following constructions and algorithms:
Noise protocol framework A collection of cryptographic handshake pat-
terns which provide building blocks to construct new secure protocols
with authenticated key agreement [47].
ChaCha20-Poly1305 The ChaCha20 stream cipher and Poly1305 authen-
ticator, used in an Authenticated Encryption with Additional Data
(AEAD) construction. This is also used in modern security protocols
such as TLS 1.3 [45]. It provides authenticity and confidentiality of
transported data.
X25519 An elliptic-curve-Diffie-Hellman (ECDH) function [4]. Compared
to other functions, it has a rather small key size and simpler require
ments regarding key validation. It is used in the key agreement proto-
col.
HKDF The HMAC-based Key Derivation Function (HKDF) is a construc-
tion to derive one or more keys from an initial secret [40]. It is used
to link all pieces of the handshake state to each other, including keys
and protocol messages. It is also ensures that the original key material
that is involved in calculations cannot be recovered.
BLAKE2 A fast cryptographic hash function, BLAKE2s [50], is used by
the HKDF and as a message authentication code (MAC).
\section{Organization}

  Chapter 2 discusses about the background knowlegde of IoT, RIOT Operating system, and the
  6LoWPAN network stack.

  Chapter 3 is about the underlying network stack that powers RIOT - GNRC.

  Chapter 4 is a thorough explaination of the wireguard protocol.

  Chapter 5 gives an evaluation of wireguard security, and how it help
  the IoT world.

  Chapter 6 gives some related works of other compressed VPN protocols, 
  and other evaluation of wireguard for constrained devices.

  Chapter 7 discusses the design and implementation of wireguard.

  Chapter 8 explains the testing of the wireguard.

  Chapter 9: conclusion and future works of this thesis.
