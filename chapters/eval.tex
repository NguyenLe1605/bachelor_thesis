\chapter{Wireguard Security and the Internet of Things}
  This chapter discusses in details security properties that Wireguard provides and both the benefits
  and drawback of adapting these features for the Internet of Things. Section \ref{iot:silence} is
  about the silence nature of Wireguard. Section \ref{iot:crypt} show the resilience of WireGuard
  against other VPN and security protocols at the transport layer. Finally, section \ref{iot:cookie} is about the details of 
  cookie mechanism. 
\section{Virtuous Silence} \label{iot:silence}
  Preventing storing any state before the authentication and transmitting responses to dubious packets
  is one of the design objectives of Wireguard. With this specific approach, Wireguard is unreachable
  from the standpoint of unauthenticated peers and network scanners. 

  The threat of mirai botnet can also be eliminated using a correct configuration of 
  Wireguard interface on the device. Mirai works by first scanning on the IoT device for
  the TCP port 23 or 2323, then brute-force logining with a random list of credentials \cite{botnet}.
  With silence nature of Wireguard, if all device's traffic is configured to go through the
  Wireguard interface except for the external UDP port, mirai won't be able to hijack the
  device for malicious reasons.

  Wireguard handshake messages are implementated without dynamic memory allocation, which is
  desirable for network protocol on constrained devices. However, this property demands the
  first initiator's message to be authenticated  from the responder, opening a risk for
  replay attack. An adversary could trick the responder into re-creating its ephemeral key
  by replay the initial handshake initiation, thus nullifying the session of the valid initiator.
  To tackle this problem, Wireguard includes an encrypted 12-byte TAI64n \cite{tai64} timestamp 
  in the first message. The responder will now discard handshake packets with timestamp that 
  is lesser than or equal to the peer's greatest received timestamp. However, a drawback of 
  this is that the increasing in size of the packet causes the fragmentation of the message.
\section{Cryptography Resilience} \label{iot:crypt}
  Adopting new VPN protocols the for IoT should be cryptographically sound, where existing
  vulnerabilites in previous protocols should not exist within the implementation. The followings
  are examples of Wireguard being resistant compare to other security protocols:
  \begin{itemize}
    \item Usage of modern cryptography algorithms prevents cryptographic attacks on TLS
    such as the Sweet32 birthday attacks \cite{sweet32}, the Bleichenbacher threat on RSA encryption \cite{bleichen}, 
    and the padding oracle attack like BEAST \cite{beast}.  Furthermore, problems with using
    constructions like MAC-then-Encrypt in IPsec \cite{ipsec_conf} are eliminated with AEAD in Wireguard.
    \item Wireguard lacks of cryptography agility \cite{boring} with fixed key exchange and cipher 
    primitives, thus preventing the  downgrade attack aiming at insecure older version of the 
    protocol like Lucky Thirteen \cite{lucky} in DTLS/TLS. Given that the underlying algorithm has 
    to be updated, the newest version must be deployed on the endpoints.
    \item The risk of vulnerabilites from complex message parsing like Heartbleed \cite{heartbleed} 
    or parser bugs in X.509 implementations \cite{x509} is mitigated due to the fixed message sizes 
    during the handshake.
    \item With the frequent re-transmission of handshake to re-generate the ephemeral key pairs, 
    attacks taking advantage of key reusing for DH exchange like Racoon attack \cite{raccoon} would be impossible 
    to launch against the protocol. Even stronger post-quantum security can be enforced with the 
    use of the pre-shared symmetric key, guaranteeing the forward secrecy of the protocol.
  \end{itemize}
\section{Cookies and Denial of Services Attack} \label{iot:cookie}
  Curve25519 point multiplication is  CPU extensive \cite{wireguard} and could lead to
  the exhaustion of battery of constrained devices despite being fast on some processor. In order to
  protect against this kind of battery-exhaustion attack, a responder could reject handshake message (either
  initiation or response from the initiator) in an under-load condition, and instead sending back
  a cookie reply message with a cookie on it. This cookie will be sent in the next handshake initation
  message from the initiator in the mac2 field.
  
  A cookie is an output of a MAC of the initiator's source IP address with a secret random value that
  changes every two minutes as a key. For the retransmission, a MAC of its message using the cookie 
  as the MAC key will be sent. As silence is one of the Wireguard virtue, the responder should not 
  send back any cookie reply message when under load for any unauthenticated packets. Thus, every 
  handshake packet incorporates a MAC in the mac1 field with the responder's public key at its MAC key. 
  This at least ensures the sending peer know who it is talking to. The MAC must always be valid 
  regardless of the under-load condition. 
  
  To prevent a man-in-the-middle capturing the plaintext cookie and creating fraudulent messages from it,
  the sending cookie is encrypted using an AEAD with extended randomize nonce \cite{irtf-cfrg-xchacha-03} 
  and the responder's public key.

  A final challenge to tackle is a denial-of-service attack on the initiator, where fraudulent cookie
  is sent to it, making the MAC computing of its message fail. Wireguard solves the problem by using
  the mac1 field of the previous handshake initiation as the additonal data field of AEAD for cookie encryption. This
  effectively binds the cookie reply to the initiation message, assuring flow of invalid cookie replies
  could not stop the initiator to authenticate the correct cookie. These three mechanism work together
  to mitigate a battery-exhaustion attack on the Wireguard-running constrained device.
