\chapter{Wireguard and 6LoWPAN}
  This chapter discusses in details security properties that Wireguard provides and both the benefits
  and drawback of implementing these features for the 6LoWPAN protocol. Section \ref{iot:silence} is
  about the silence nature of Wireguard. Section \ref{iot:crypt} show the resilience of WireGuard
  against other VPN and transport layer security protocols. Section \ref{iot:cookie} is about the details of 
  cookie mechanism. Section \ref{iot:perist} is on the optional chatty features of Wireguard.
  Finally, section \ref{iot:mem} discusses the 6LoWPAN fragmentation impact on the Wireguard messages.
\section{Silence is a virtue} \label{iot:silence}
\section{Cryptography Resilience} \label{iot:crypt}
  The design decisions of Wireguard allow the protocol itself and its implementation resistant to
  solve many problems of the other traditional VPN protocols:
  \begin{itemize}
    \item Usage of modern cryptography algorithms prevents cryptographic attacks on TLS
    such as the Sweet32 birthday attacks \cite{sweet32}, the Bleichenbacher threat on RSA encryption \cite{bleichen}, 
    and the padding oracle attack like BEAST \cite{beast}.  Furthermore, problems with using
    constructions like MAC-then-Encrypt in IPsec \cite{ipsec_conf} are eliminated with AEAD in Wireguard.
  \end{itemize}
\section{Cookies and Denial of Services Attack} \label{iot:cookie}
\section{Stateful firewall} \label{iot:perist}
\section{Message \& Fragmentation} \label{iot:mem}
